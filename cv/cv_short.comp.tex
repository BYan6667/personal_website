%%% A template to produce a nice-looking Curriculum Vitae.
%%% Kieran Healy <kjhealy@gmail.com>
%%% Most recent version is at http://kjhealy.github.com/kjh-vita
%%%
%%% ------------------------------------------------------------------------
%%% Requirements (should be included in a modern tex distribution):
%%% ------------------------------------------------------------------------
%%% xelatex
%%% fontspec.sty
%%% hyperrref.sty
%%% xunicode.sty
%%% color.sty
%%% url.sty
%%% fancyhdr.sty
%%%
%%% ------------------------------------------------------------------------
%%% Optional
%%% ------------------------------------------------------------------------
%%% git
%%% vc.sty
%%% revnum.sty
%%% Fonts
%%%
%%% ------------------------------------------------------------------------
%%% Note
%%%------------------------------------------------------------------------
%%% Because this is a hand-tweaked file, be on the look out for \medksip,
%%% \bigskip and \newpage commands here and there, which are used to balance
%%% the layout or avoid widows & orphans, etc. You should of course add or
%%% remove these as needed.
%%%------------------------------------------------------------------------

\documentclass[11pt]{article}
\renewcommand{\arraystretch}{1.3}

%%%------------------------------------------------------------------------
%%% Metadata
%%%------------------------------------------------------------------------

%% Change as needed. Or just add me as a coauthor. Only some of these are
%% used below in the hyperref declaration and address banner section.
\def\myauthor{Alexander Rush}
\def\mytitle{Vita}
\def\mycopyright{\myauthor}
\def\mykeywords{}
\def\mybibliostyle{plain}
\def\mybibliocommand{}
\def\mysubtitle{}
\def\myaffiliation{cornell}
\def\myaddress{Computer Science Department}
\def\myemail{arush@cornell.edu}
\def\myweb{http://rush-nlp.com}
\def\myphone{(215) 317-8089}
\def\myfax{srush\_nlp}
\def\myversion{}
\def\myrevision{}


\def\myaffiliation{Cornell University}
\def\myauthor{Alexander Rush}
\date{} % not used (revision control instead)
\def\mykeywords{Rush, Alexander, Alexander Rush, Vita, CV, Resume, Computer Science}


%%%------------------------------------------------------------------------
%%% Git version tracking
%%%------------------------------------------------------------------------

%% If you don't use git or the vc package (from CTAN), comment this out.
%% If you comment it out, be sure to remove the \rfoot comment below, too.
% \input{vc}

%%%------------------------------------------------------------------------
%%% Required style files
%%%------------------------------------------------------------------------
\usepackage{url,fancyhdr}
%%\usepackage{revnum} % for reverse-numbered publications (revnumerate environment) if needed.

%% needed for xelatex to work
% \usepackage{fontspec}
% \usepackage{xunicode}

%% color for the links
\usepackage[usenames,dvipsnames]{xcolor}

%% hyperlinks
\usepackage[
% xetex,
colorlinks=true,
	urlcolor=BlueViolet,
	plainpages=false,
  	pdfpagelabels,
  	bookmarksnumbered,
  	pdftitle={\mytitle},
  	pagebackref,
  	pdfauthor={\myauthor},
  	pdfkeywords={\mykeywords}
  	]{hyperref}

%%%------------------------------------------------------------------------
%%% Document
%%%------------------------------------------------------------------------
\begin{document}

%% Choose fonts for use with xelatex
%% Minion and Myriad are widely available, from Adobe.
%% Pragmata is available to buy at http://www.fsd.it/fonts/pragma.htm
%% and is worth every penny. Any good monospace font will work fine, though.
%% Consolas or inconsolata are good alternatives.
% \setromanfont[Mapping={tex-text},Numbers={OldStyle},Ligatures={Common}]{Minion Pro}
% \setsansfont[Mapping=tex-text,Colour=AA0000]{Myriad Pro}
% \setmonofont[Mapping=tex-text,Scale=0.9]{Inconsolata}


%%%------------------------------------------------------------------------
%%% Local commands
%%%------------------------------------------------------------------------

%% Marginal header
%% Note: as the document goes on you may need to introduce a (gradually increasing)
%% \vspace element to keep the marginal header pleasingly aligned with the first
%% item in the body text. Like this: \marginhead{ {\vskip 0.4em}Grants}, or
%% \marginhead{ {\vskip 0.8em}Service}. Experiment as needed.
\newcommand{\marginhead}[1]{\marginpar{\textsf{ {\footnotesize\vspace{-1em}\flushright #1}}}}


%% custom ampersand (font consistent with the one chosen above)
\newcommand{\amper}{ {\fontspec[Scale=.95,Colour=AA0000]{Minion Pro Medium}\selectfont\&\,}}

%% No bullets on labels
\renewcommand{\labelitemi}{~}

%% Custom hanging indent for vita items
\def\ind{\hangindent=1 true cm\hangafter=1 \noindent}
%\def\ind{\hangindent=18pt\hangafter=1 \noindent}
\def\labelitemi{~}
\renewcommand{\labelitemii}{~}

%%%------------------------------------------------------------------------
%%% Page layout
%%%------------------------------------------------------------------------
\pagestyle{fancy}
\renewcommand{\headrulewidth}{0pt}
\fancyhead{}
\fancyfoot{}
\rhead{ {\scriptsize\thepage}}

%% git revision control footer
%\rfoot{\texttt{\scriptsize \VCRevision\ on \VCDateTEX}} % git revision info inserted via external script -- see docs for vc package for details. comment out this line if you're not using vc, and also remove the \input{vc} line above.

%%%------------------------------------------------------------------------
%%% Address and contact block
%%%------------------------------------------------------------------------
\begin{minipage}[t]{2.95in}
   \flushright {\footnotesize \href{http://rush-nlp.com}{Cornell University} \\ Cornell Tech, \\ \vspace{-0.05in} New York, NY }


\end{minipage}
\hfill
%\begin{minipage}[t]{0.0in}
% dummy (needed here)
%\end{minipage}
\hfill
\begin{minipage}[t]{1.7in}
  \flushright \footnotesize Phone: \myphone \\
  {\scriptsize  \texttt{\href{mailto:\myemail}{\myemail}}} \\
  {\scriptsize  \texttt{\href{\myweb}{\myweb}}} \\
  {\scriptsize  \texttt{\href{http://twitter.com/\myfax}{@\myfax}}} \\
\end{minipage}


\medskip

%% Name
\noindent{\Large {\textsc{alexander m. rush}}}
\reversemarginpar

\medskip


\marginhead{Appointment}
\noindent\emph{Harvard University School of Engineering and Applied Sciences \vspace{0.01in}}\\
\ind 2015-.  Assistant Professor of Computer Science

\medskip
\noindent\emph{Facebook Artificial Intelligence Research Lab \vspace{0.01in}}\\
\ind 2015.  Post-Doctoral Fellowship\\
\ind Advisor: Yann LeCunn

\bigskip

\marginhead{Education}


\noindent\emph{Massachusetts Institute of Technology \vspace{0.01in}}\\
\ind 2009-2014.  Ph.D, Computer Science.\\
\ind Advisor: Michael Collins\\
\ind Dissertation: \emph{Relaxation Methods for Natural Language Decoding}. %\vspace{-0.1in}

\medskip
\noindent\emph{Columbia University\vspace{0.02in}}\\
\ind 2011-2014. Visiting Scholar, Department of Computer Science.


\medskip
\noindent\emph{Harvard University\vspace{0.02in}}\\
\ind 2007. B.A., Computer Science. (Magna Cum Laude With Highest Honors.)

\bigskip

%% Publications

\marginhead{ {\vskip 0.4em}Grants and \newline Awards}
\medskip

\hspace{-1cm} \begin{tabular}{lp{11.5cm}}
2012 & Best Paper Award, North American Association of Computational Linguistics. \\
2010 & Best Paper Award, Empirical Methods in Natural Language Processing. \\
2009 & Graduate Research Fellow, National Science Foundation.  \\
2006 & Outstanding Undergraduate Award Finalist, Computing Research Association. \\

\end{tabular}
\bigskip

\marginhead{ {\vskip 0.3em} Recent Publications}
\medskip



\ind Udit Gupta, Brandon Reagen, Lillian Pentecost, Marco Donato, Thierry Tambe, Alexander M. Rush, Gu-Yeon Wei, David Brooks. \emph{\href{ None }{ MASR: A Modular Accelerator for Sparse RNNs.} }\emph{ PACT 2019 }

\medskip



\ind Joe Davison, Joshua Feldman and Alexander Rush. \emph{\href{ None }{ Commonsense Knowledge Mining from Pretrained Models.} }\emph{ EMNLP 2019 }

\medskip



\ind Zachary Ziegler, Yuntian Deng and Alexander Rush. \emph{\href{ https://arxiv.org/abs/1909.01496 }{ Neural Linguistic Steganography.} }\emph{ EMNLP 2019 }

\medskip



\ind Yoon Kim,  Chris Dyer, Alexander M. Rush. \emph{\href{ https://www.aclweb.org/anthology/P19-1228/ }{ Compound Probabilistic Context-Free Grammars for Grammar Induction.} }\emph{ ACL 2019 }

\medskip



\ind Gehrmann S, Strobelt H, Krueger R, Pfister H, and Alexander M. Rush. \emph{\href{ https://arxiv.org/abs/1907.10739 }{ Visual Interaction with Deep Learning Models through Collaborative Semantic Inference.} }\emph{ InfoVis 2019 }

\medskip



\ind Jiawei Zhou, Alexander M. Rush. \emph{\href{ https://www.aclweb.org/anthology/P19-1503 }{ Simple Unsupervised Summarization by Contextual Matching.} }\emph{ ACL 2019 }

\medskip



\ind Sebastian Gehrmann, Hendrik Strobelt, Alexander M Rush. \emph{\href{ https://arxiv.org/abs/1906.04043 }{ GLTR: Statistical Detection and Visualization of Generated Text.} }\emph{ ACL Demo 2019 (Best Demo Nominee) }

\medskip



\ind Yoon Kim, Alexander M. Rush, Lei Yu, Adhiguna Kuncoro, Chris Dyer, Gabor Melis. \emph{\href{ https://arxiv.org/pdf/1904.03746.pdf }{ Unsupervised Recurrent Neural Network Grammars.} }\emph{ NAACL 2019 }

\medskip



\ind Adji B. Dieng, Yoon Kim, Alexander M. Rush, David M. Blei. \emph{\href{ https://arxiv.org/pdf/1807.04863.pdf }{ Avoiding Latent Variable Collapse With Generative Skip Models.} }\emph{ AISTATS 2019 }

\medskip



\ind Fritz Obermeyer, Eli Bingham, Martin Jankowiak, Justin Chiu, Neeraj Pradhan, Alexander Rush, Noah Goodman. \emph{\href{ https://arxiv.org/pdf/1902.03210.pdf }{ Tensor Variable Elimination for Plated Factor Graphs.} }\emph{ ICML 2019 }

\medskip











































































\ind Alexander M. Rush and Slav Petrov, \emph{\href{http://people.csail.mit.edu/srush/vine-paper.pdf}{Vine Pruning for Efficient Multi-Pass Dependency Parsing}}. Proceedings of NAACL 2012. [Best Paper Award]

%\textbf{Received Best Paper Award}

\medskip


\ind Terry Koo, Alexander M. Rush, Michael Collins, Tommi Jaakkola, and David Sontag. \emph{\href{http://people.csail.mit.edu/maestro/papers/koo10mstdd.pdf}{Dual Decomposition for Parsing with Non-Projective Head Automata}}. Proceedings of EMNLP 2010. [Best Paper Award]


\vspace{0.1in}

\noindent\emph{Patents \vspace{0.01in}}

% \medskip

\ind Techniques for discriminative dependency parsing (Google). Slav Petrov, Alexander M. Rush, 2015.
\medskip

\ind Efficient parsing with structured prediction cascades (Google). Slav Petrov, Alexander M. Rush, 2013
\medskip

%% Publications
\ind  Determining user affinity towards applications on a social networking website (Facebook., Thomas S. Whitnah, Alexander M. Rush, Ding Zhou, Ruchi Sangvhi, 2010.


\marginhead{ {\vskip 0.4em}Industry }

\bigskip

\ind  Lead Engineer (Platform Team), {\sl Facebook}, June 2007 -- August 2009, Palo Alto, CA.

\bigskip

%% Publications


\marginhead{ {\vskip 0.4em}Open-Source}
\bigskip



\medskip \ind \href{ https://github.com/harvardnlp/pytorch-struct }{ Pytorch Struct } - structured prediction in pytorch
% \begin{itemize}
% \item
% \end{itemize}



\medskip \ind \href{ http://steganography.live }{ SteganographyLive } - neural stegonagraphy
% \begin{itemize}
% \item
% \end{itemize}



\medskip \ind \href{ https://github.com/harvardnlp/urnng }{ Unsupervised Recurrent Neural Network Grammars } - unsupervised learning of recurrent neural network grammars
% \begin{itemize}
% \item
% \end{itemize}



\medskip \ind \href{ https://github.com/harvardnlp/neural-template-gen }{ Learning Neural Templates for Text Generation } - neural templates for text generation
% \begin{itemize}
% \item
% \end{itemize}









































\end{document}
